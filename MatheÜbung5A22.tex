\documentclass[11pt, a4paper]{article}


\usepackage[utf8]{inputenc}
\usepackage{textcomp}
\usepackage[dvipsnames]{xcolor}
\usepackage[ngermanb]{babel}
\usepackage{amsmath,amscd,amssymb}
\usepackage{enumerate}
\usepackage{mathtools}
\usepackage[headsep=.5cm,headheight=2cm,left=2.5cm,right=2.5cm,top=3cm]{geometry}
\usepackage{fancyhdr}

\pagestyle{fancy}
\fancyhf{}
\rhead{Nils Urban}
\lhead{Übung \uebungsnr}
\cfoot{\thepage}

\parindent0cm

\newcommand{\uebungsnr}{5}
\newcommand{\aufgabe}[1]{{\center\large\bfseries Gruppenaufgabe #1 \\}}


\begin{document}
\aufgabe{22}
\begin{enumerate}[a)]
	\item Es handelt sich um eine Äquivalenzrelation. Jedes Buch wurde vom selben Autor geschrieben, wie das Buch selbst. Wenn ein Buch $a \in M$ vom selben Autor wie ein Buch $b \in M$ geschrieben wurde, so wurde das Buch $b$ auch vom selben Autor geschrieben wie Buch $a$. Wenn Bücher $a$ und $b$ den selben Autor haben, und Buch $b$ und ein weiteres Buch $c \in M$ ebenfalls den selben Autor haben, so haben auch Bücher $a$ und $c$ den selben Autor. Die Relation ist reflexiv, symmetrisch und transitiv. Somit sind alle Eigenschaften einer Äquivalenzrelation erfüllt.
	\item Es handelt sich nicht um eine Äquivalenzrelation, da die Relation nicht transitiv ist.\\ Es werden drei Bücher $a, b, c \in M$ betrachtet. Zudem wird angenommen, dass nur zwei Personen jemals in der Bibliothek Bücher ausgeliehen haben. Eine Person $p_1$ hat ausschließlich Bücher $a$ und $b$ mal ausgeliehen. Eine andere Person $p_2$ hat ausschließlich die Bücher $b$ und $c$ mal ausgeliehen. In diesem Fall gibt es keine Person, die sowohl Buch $a$ als auch Buch $c$ mal ausgeliehen hat. Dies wäre für die Transitivität der Relation jedoch Vorraussetzung.
	\item Es handelt sich nicht um eine Äquivalenzrelation, da die Relation nicht symmetrisch ist. Für jede zwei Bücher $x$ und $y$ aus $M$ müsste für die Symmetrie gelten: $xRy \rightarrow yRx$. Dies würde im Kontext bedeuten, dass ein Buch, welches mehr Seiten als ein anderes Buch hat, gleichzeitig weniger Seiten als dieses andere Buch besitzt. Das ist unmöglich.
\end{enumerate}
\end{document}
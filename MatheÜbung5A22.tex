\documentclass[11pt, a4paper]{article}


\usepackage[utf8]{inputenc}
\usepackage{textcomp}
\usepackage[dvipsnames]{xcolor}
\usepackage[ngermanb]{babel}
\usepackage{amsmath,amscd,amssymb}
\usepackage{enumerate}
\usepackage{mathtools}
\usepackage[headsep=.5cm,headheight=2cm,left=2.5cm,right=2.5cm,top=3cm]{geometry}
\usepackage{fancyhdr}

\pagestyle{fancy}
\fancyhf{}
\rhead{Nils Urban}
\lhead{Übung \uebungsnr}
\cfoot{\thepage}

\parindent0cm

\newcommand{\uebungsnr}{5}
\newcommand{\aufgabe}[1]{{\center\large\bfseries Gruppenaufgabe #1 \\}}


\begin{document}
\aufgabe{22}
\begin{enumerate}[a)]
	\item Es handelt sich um eine Äquivalenzrelation
	\item Es haldelt sich nicht um eine Äquivalenzrelation, da die Relation nicht transitiv ist. Werden drei Bücher $a, b, c \in M$ betrachtet
\end{enumerate}
\end{document}
\documentclass[11pt, a4paper]{article}


\usepackage[utf8]{inputenc}
\usepackage[ngermanb]{babel}
\usepackage{amsmath,amscd,amssymb}
\usepackage{enumerate}
\usepackage{mathtools}
\usepackage[headsep=.5cm,headheight=2cm,left=2.5cm,right=2.5cm,top=3cm]{geometry}
\usepackage{fancyhdr}

\pagestyle{fancy}
\fancyhf{}
\lhead{Übung \aufgabennr}
\cfoot{\thepage}

\parindent0cm

\newcommand{\aufgabennr}{32}


\begin{document}

{\center\large\bfseries Übung \aufgabennr\\}

\begin{enumerate}[(a)]
    \item {\bfseries Aufgabe:\\}
    Bestimmen sie $f \circ g$.\\
    {\bfseries Lösung:\\}
    $f \circ g: \mathbb{Z} \rightarrow \{1,2,3,...,p - 1\}, x \mapsto 10$, wenn $\exists y \in \mathbb{N}: x = 3*y + 1$, sonst 1\\
    \item {\bfseries Aufgabe:\\}
    Geben sie den Wertebereich an, sodass die Funktion surjektiv ist.\\
    {\bfseries Lösung:\\}
    Eine Funktion ist surjektiv, wenn jedes Element des Wertebereichs ein nicht leeres Urbild hat.\\
    Das Bild einer Zahl, für die $\exists y \in \mathbb{N}: x = 3*y + 1$ gilt, ist immer 10, da wenn $g$ auf diese Zahl angewendet wird, als Bild von $g$ 10 weiter an $f$ gegeben wird und $2^{10}\ mod\ 13 = 10$ gilt.\\
    Das Bild jeder anderen Zahl ist 1, da wenn eine Zahl durch 3 teilbar ist, das Bild über $g$ als 0 weitergegeben wird und $2^0 = 1$, und da für jede andere Zahl 12 weitergegeben wird und $2^{12}\ mod\ 13 = 1$ gilt.\\
    Dementsprechend ist $\{1, 10\}$ der Wertebereich, der die Funktion surjektiv macht.
    \item {\bfseries Aufgabe:\\}
    Bestimmen sie $g \circ f$.\\
    {\bfseries Lösung:\\}
    $g \circ f: \{0,2,3,...,p-1\} \rightarrow \{0,2,3,...,p-1\}, x \mapsto \begin{cases} 0, & \text{wenn } 3 \mid (a^x \mod p) \\ 10, & \text{wenn } \exists y \in \mathbb{N} : a^x = 3 \cdot y + 1 \mod p \\ 12, & \text{sonst} \end{cases}$\\
    \item {\bfseries Aufgabe:\\}
    Bestimmen sie $(g \circ f)^{-1}(0)$.\\
    {\bfseries Lösung:\\}
    $(g \circ f)^{-1}(0) = \{4, 5, 6, 8\}$\\
    \item {\bfseries Aufgabe:\\}
    Warum ist der Vorschlag für die Verschlüsselung ungeeignet?\\
    {\bfseries Lösung:\\}
    Wenn man die neue Funktion $g \circ (f \circ g)$ auf eine beliebige ganze Zahl anwendet, muss zuerst $(f \circ g)$ und danach noch $g$ angewendet werden.
    Wie in (b) erklärt ist das Bild für beliebige ganze Zahlen, auf die die Funktion angewendet werden soll, Element aus $\{1, 10\}$, also 1 oder 10.
    Wendet man $g$ auf 1 an, so ist die Bedingung $\exists y \in \mathbb{N} : x = 3 * y + 1$ erfüllt, da $1 = 3*0 + 1$ gilt. Das Bild ist also 10.
    Wendet man $g$ auf 10 an, so ist die Bedingung $\exists y \in \mathbb{N} : x = 3 * y + 1$ erfüllt, da $10 = 3*3 + 1$ gilt. Das Bild ist auch also 10.
    Die Funktion $g \circ (f \circ g)$ bildet jede ganze Zahl also auf 10 ab. Da sich die Gesprächspartner anhand des Bilds authentifizieren wollen, ist das vorgeschlagene Verfahren ungeeignet.\\
\end{enumerate}

\end{document}
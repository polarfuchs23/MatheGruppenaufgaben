\documentclass[11pt, a4paper]{article}


\usepackage[utf8]{inputenc}
\usepackage[ngermanb]{babel}
\usepackage{amsmath,amscd,amssymb}
\usepackage{enumerate}
\usepackage{mathtools}
\usepackage[headsep=.5cm,headheight=2cm,left=2.5cm,right=2.5cm,top=3cm]{geometry}
\usepackage{fancyhdr}
\usepackage{ stmaryrd }

\pagestyle{fancy}
\fancyhf{}
\lhead{Übung \aufgabennr}
\cfoot{\thepage}

\parindent0cm

%%%%%%%%%%%%%%%%%%%%%%%%%%%%%%%%%%%%%%%%%%%%%%%%%%%%%%%%%%%%%%%%%%%%%%%%%%%%%%%
%%% Bis hier muss nichts verändert werden
%%%%%%%%%%%%%%%%%%%%%%%%%%%%%%%%%%%%%%%%%%%%%%%%%%%%%%%%%%%%%%%%%%%%%%%%%%%%%%%

%%%%%%%%%%%%%%%%% Hier bitte die X ersetzen

\newcommand{\aufgabennr}{11}


\begin{document}
\begin{enumerate}
    \setcounter{enumi}{40}
    \item {\bfseries Aufgabe:\\}
    Zeigen sie folgende Aussagen direkt:\\
    \begin{enumerate}
        \item Seien $a, b \in \mathbb{R}$ mit $0 < a, 0 < b$. Z.z.: $\frac{a}{b} + \frac{b}{a} \geq 2$\\
        \item Zeigen Sie, dass sich jede ungerade Primzahl als Differenz zweier Quadratzahlen darstellen
        lässt.\\
    \end{enumerate}
    {\bfseries Lösung:\\}
    \begin{enumerate}
        \item \dots
        \item Sei $x$ eine beliebige ungerade Primzahl.
        \begin{align*}
            \Rightarrow& x = 1 \cdot x &&|~\text{x ist ungerade}\\
            \Rightarrow& \exists a \in \mathbb{N}: x = 1 \cdot (2 \cdot a + 1) &&|~\text{Sei b = a + 1}\\
            \Rightarrow& \exists a, b \in \mathbb{N}: x = 1 \cdot (a + b) &&|~\text{da b = a + 1}\\
            \Rightarrow& \exists a, b \in \mathbb{N}: x = (b - a) \cdot (a + b) &&|~\text{Binomische Formel}\\
            \Rightarrow& \exists a, b \in \mathbb{N}: x = a^2 - b^2 &&\square\\
        \end{align*}
    \end{enumerate}
    \item {\bfseries Aufgabe:\\}
    Beweisen Sie die folgenden Sätze indirekt (mittels Widerspruch):\\
    \begin{enumerate}
        \item Wenn n eine natürliche Zahl ist, dann ist $n + (n + 1) + (n + 2)$ durch $3$ teilbar.\\
        \item $\forall a, b \in \mathbb{R}^+: \sqrt[2]{ab} \leq \frac{a+b}{2}$\\
    \end{enumerate}
    {\bfseries Lösung:\\}
    \begin{enumerate}[(a)]
        \item \dots\\
        \item Annahme: $\sqrt[2]{ab} > \frac{a+b}{2}$\\
        \begin{align*}
            \Rightarrow& ab > (\frac{a+b}{2})^2 &&|~\text{Binomische Formel}\\
            \Rightarrow& ab > \frac{a^2+2\cdot ab+b^2}{4} &&|~\text{Bruch auflösen}\\
            \Rightarrow& ab > \frac{a^2 + b^2}{4} + \frac{ab}{2} &&|~ab = 2\cdot \frac{ab}{2}\\
            \Rightarrow& \frac{ab}{2} + \frac{ab}{2} > \frac{a^2 + b^2}{4} + \frac{ab}{2} &&|~-\frac{ab}{2}\\
            \Rightarrow& \frac{2ab}{4} > \frac{a^2 + b^2}{4} &&|~\cdot 4\\
            \Rightarrow& 2ab > a^2 + b^2
        \end{align*}
        Fall 1: $a=b$\\
        \begin{align*}
            &2ab > a^2 + b^2 &&|~\text{Fallbedingung}\\
            \Rightarrow& 2a\cdot a > a^2 + a^2\\
            \Rightarrow& 2a^2 > 2a^2 && \lightning \\
        \end{align*}
        Fall 2: Da $a$ und $b$ vertauschbar sind, sind die Fälle $a > b$ und $b > a$ identisch. Es sei $a > b$:
        \begin{align*}
            &2ab > a^2 + b^2 &&\\
            \Rightarrow& 2ab > ab + a \cdot (a-b) + ab - b \cdot (a-b) &&|~\text{Distributivg.}\\
            \Rightarrow& 2ab > 2ab + a \cdot (a-b) - b \cdot (a-b) &&|~\text{Distributivg.}\\
            \Rightarrow& 2ab > 2ab + (a-b) \cdot (a-b) &&\lightning \text{, da }a>b\text{ und daher }(a-b)>0\\
        \end{align*}
        $\Rightarrow$ Da die Annahme widerlegt wurde, gilt die Aussage.\\
    \end{enumerate}
    \item {\bfseries Aufgabe:\\}
    Beweisen Sie mittels Kontraposition:\\
    \begin{enumerate}
        \item Sei $x \in \mathbb{Z}$: Wenn $x^2 - 6x + 5$ gerade ist, dann ist $x$ ungerade.\\
        \item Seien $f : A \mapsto B, g : B \mapsto C$ Abbildungen. Wenn $g \circ f$ bijektiv ist, dann ist $f$ injektiv und $g$
        surjektiv.\\
    \end{enumerate}
    {\bfseries Lösung:\\}
    \begin{enumerate}[(a)]
        \item \dots\\
        \item Es wird die Kontraposition zur Aussage, Wenn $f$ nicht injektiv oder $g$
        nicht surjektiv ist, dann ist $g \circ f$ nicht bijektiv.\\
        $f$ ist nicht injektiv.\\
        \begin{align*}
            \Rightarrow& \exists a_1, a_2 \in A \exists b_1 \in B: f(a_1) = b_1 = f(a_2) &&|~g\text{ auf }b_1 \text{ anwenden}\\
            \Rightarrow& \exists a_1, a_2 \in A \exists b_1 \in B \exists c_1 \in C: f(a_1) = b_1 = f(a_2) \land g(b_1) = c_1 &&|~g \circ f \text{ anwenden}\\
            \Rightarrow& \exists a_1, a_2 \in A \exists c_1 \in C: g \circ f(a_1) = c_1 = f(a_2) &&\\
            \Rightarrow& g \circ f\text{ ist nicht injektiv und daher nicht bijektiv.}
        \end{align*}
        $g$ ist nicht surjektiv.\\
        \begin{align*}
            \Rightarrow& \exists c_1 \in C \nexists b_1 \in B: g(b_1) = c_1 &&|~f \text{anwenden}\\
            \Rightarrow& \nexists a_1 \nexists b_1 \exists c_1: f(a_1) = b_1 \land g(b_1) = c_1 &&|~g \circ f \text{anwenden}\\
            \Rightarrow& \nexists a_1 \exists c_1: g \circ f(a_1) = c_1 \\
            \Rightarrow& g \circ f\text{ ist nicht surjektiv und daher nicht bijektiv.}
        \end{align*}
        $\Rightarrow$ Da die Kontraposition zur Aussage gilt, gilt auch die Aussage.\\
    \end{enumerate}
    \item {\bfseries Aufgabe:\\}
    Sei $M = {1, 2, 3, \dots, 2n}$ und $A \subseteq M$ mit $\#A = n + 1$. Zeigen Sie, dass gilt:\\
    $\exists a, b \in A: a \neq b \land a|b$\\
    {\bfseries Lösung:\\}
    Da es unter den natürlichen Zahlen bis $2n$ $n$ gerade und $n$ ungerade Zahlen gibt und $1$ nicht in $A$ genommen werden kann, da sie alle anderen
    natürlichen Zahlen teilt, muss es mindestens zwei gerade Zahlen in $A$ geben. Diese haben entweder nur gerade Faktoren oder einen geraden und einen
    ungeraden Faktor. Zahlen die nur gerade Faktoren haben sind zweier Potenzen. Da sich alle zweier Potenzen gegenseitig teilen, können nicht zwei Zahlen
    aus ausschließlich geraden Zahlen gewählt werden. Es muss eine gerade Zahl mit einem ungeradem Teiler gewählt werden. Dieser Teiler kann dann nicht mehr
    in $A$ sein. Es muss also mindestens eine gerade Zahl mehr gewählt werden. Für diese Zahl gelten die selben Regeln wie zuvor. Zusätlich dürfen die
    bereits gestrichenen ungeraden Zahlen kein Teiler der Zahl sein. Die neue gerade Zahl muss also einen noch nicht gestrichenen ungeraden Faktor haben.
    Auch dieser muss dann gestrichen werden und wie zuvor immer weiter durch eine gerade Zahl ersetzt werden. Sind so alle ungeraden Zahlen bis $n$ gestrichen
    worden, kann keine gerade Zahl mehr gewählt werden, die nur nicht gestrichene Teiler hat, da die Zahl kleiner oder gleich $2n$ sein muss, $n$ also der gößte
    ungerade Teiler sein kann. Es muss also eine Zahl gewählt werden, die einen Teiler hat, der bereits in $A$ ist. Es lässt sich daher keine Menge bilden
    für die die Aussage wahr ist. 
\end{enumerate}
\end{document}